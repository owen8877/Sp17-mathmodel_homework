\documentclass[a4paper]{article}
\usepackage{geometry}
\geometry{left=3.5cm,right=3.5cm,top=3.5cm,bottom=3.5cm}
\usepackage{amsmath, amssymb}
\usepackage{graphicx}
\usepackage{subfigure}
\usepackage{pdfpages}
\usepackage{multirow}

\begin{document}
\part{3-1}
First we use \emph{scatteredInterpolant()} in MATLAB to convert scattered data into grid data.
\section{How to find max points}
Select reasonable parameter $b$, so that we aim to find the points that are largest in its $b*b$ neighbourhood, so as to say
$$ \rho_{i, j} \ge \rho_{i+k, j+l}, for \forall |k|, |l| \le b$$

Have found such peak points, we have to measure their pollution rate. 
Consider the pollution follows Gauss distribution; so the pollution it causes is related to its pollution density and the 2nd derivative.
$$ pollution rate \quad proportion to \quad \rho^2 / \Delta \rho $$
Take the first half contributions and we get the heavily-polluted origins.

\section{simulation}
the first equation is quite easy
$$ = \Delta \rho * \kappa $$
if we take the pollution into account
$$ = \Delta \rho * \kappa + some pollution (use arbitrary alphabet) if (x, y) \in peakCoordinate $$
if we take height into account
$$ = \Delta \rho * \kappa + \Delta h * \tau * \rho + some pollution (use arbitrary alphabet) if (x, y) \in peakCoordinate $$
$\tau$ is the parameter representing the speed term of water
\end{document}