\documentclass[a4paper]{article}
\usepackage{geometry}
\geometry{left=3.5cm,right=3.5cm,top=3.5cm,bottom=3.5cm}
\usepackage{amsmath, amssymb}
\usepackage{graphicx}
\usepackage{subfigure}
\usepackage{pdfpages}
\usepackage{multirow}

\usepackage{xeCJK}
\setmainfont{Times New Roman}
\setCJKmainfont[BoldFont=SimHei,ItalicFont=KaiTi]{SimSun}

\usepackage{indentfirst}
\setlength{\parindent}{2em}

\usepackage{fancyhdr}
\pagestyle{fancy}
\usepackage{lastpage}
\rhead{}
\lhead{}
\cfoot{\thepage{}}
\renewcommand{\headrulewidth}{0pt}
\renewcommand{\figurename}{图}
\renewcommand{\tablename}{表}
\renewcommand{\abstractname}{摘要}
\renewcommand{\contentsname}{\CJKfamily{SimHei} 目录}

\headheight 14pt

\usepackage{float}

\renewcommand\baselinestretch{1.2}

\begin{document}

%\begin{titlepage}
%\includepdf[pages=2-3,offset=0cm 0cm]{title.pdf}
%\end{titlepage}

\begin{Huge}
	\centering{\textbf{}}
\end{Huge}
\begin{large}
	\begin{flushright}
		
	\end{flushright}
\end{large}
\ \ \\\\

\begin{abstract}
\textit{}
\end{abstract}

\newpage

\tableofcontents

\newpage

\part{引言}
土壤是生态环境的重要组成部分, 也是人类赖以生存的重要自然资源。然而,随着工农业的发展,土壤污染问题越来越突出,尤其是重金属污染,对于土壤的污染尤其严重。
所谓土壤重金属污染是指由于人类的活动将重金属带入土壤中,致使土壤重金属含量明显高于其自然背景值,并造成生态破坏和环境质量恶化的现象。
当前,随着全球经济一体化的发展,我国也加快了工业化与城市化的发展进程,城市的农业集约化程度不断提高。
生产过程中的固体废物不断向土壤表面堆放和倾倒,有害废水不断向土壤中渗透,汽车排放的废气,大气中的有害气体及飘尘不断随雨水降落在土壤中。
这导致水土资源快速恶化和萎缩,土壤利用强度日益加大。   \\
城市土壤供应着城市绿色植物的生长介质和养分,是城市污染物重要的源和汇,是土壤微生物的栖息地和能量来源,直接影响到城市的生态环境质量。
重金属污染物主要包括汞、镐、铅、铬、锌、铜、镍、钴、锡以及类金属砷等,作为一种持久性有毒物质,进入土壤系统。
从而会使城市土壤的各种性质发生了变化,引起土壤的组成结构和功能发生变化,微生物活动受到抑制,有害物质或分解产物在土壤中逐渐积累,间接被人体吸收,危害人体健康。
因此,土壤重金属污染已成为全球面临的一个严重环境问题。   \\
国土资源部统计表明,全国320个严重污染区约有548万hm2土壤,大田类农产品污染超标面积占污染区农田面积的20\%,其中重金属污染占80\%,粮食中重金属镉、砷、铬、铅、汞等的超标率占10\%。
在城市,土地污染问题也十分严重。例如,被公认为城市环境质量优良的公园却也存在着严重的土壤重金属污染。
而且我国每年有 1200 万吨粮食遭到重金属污染,直接经济损失超过 200 亿元。
从我国的土地资源情况来看,人均土地资源远远低于世界平均水平,仅及世界人均占有量的1/3。我国耕地人均只有0.1公顷,为世界人均耕地的27.7%。
由此可见,目前我国土地重金属污染非常严重,而且还影响了到人们的食品安全乃至人们的身体健康。
因而如何有效地治理土地重金属污染问题,改良土壤状况,是目前生态环境保护的一个重要主题。    \\
本文以某市不同功能区的表层土壤为研究对象,对土壤中重金属 As、Cd、Cr、
Cu、Hg、Ni、Pb、Zn 的浓度进行测定所获数据进行分析,评价土壤中重金属的污染程度,分析污染的主要原因并定位污染源,



\part{重金属在不同区域的分布情况}





\begin{equation}
\frac{\partial \rou}{\partial t} = \kappa \Delta \rou + \tau \Delta u
\end{equation}